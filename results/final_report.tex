\documentclass[11pt]{article}

\usepackage[margin=1in]{geometry}
\usepackage{graphicx}
\usepackage{booktabs}
\usepackage{float}
\usepackage{hyperref}

\title{\textbf{Columbus, Ohio Weather in Practice}\\DSCI 510 Final Project Report}
\author{Yao Lin}
\date{December 2025}

\begin{document}
\maketitle

\section{Project Title and Team Members}
\textbf{Project:} Columbus, Ohio Weather in Practice\\
\textbf{Team Member:} Yao Lin

\section{Research Question and Short Description}
 From July to November 2025 in Columbus, Ohio, how did daily temperature and precipitation evolve, and how did the city shift from cooling-driven conditions to heating-driven conditions?

This project builds a decision-ready, non-technical view of recent local weather conditions using daily climatological summaries. 
The analysis

(1) temperature trends and variability, 

(2) precipitation frequency and intensity

(3) cooling/heating demand using degree-day metrics .

\section{Data Description}

\subsection{What Data Were Collected?}
The dataset is a daily climatological table for the Columbus area (ThreadEx daily summaries) covering:
 Date, Month
 
 Temperature (F): Maximum, Minimum, Average
 
 Departure (F): daily temperature departure from normal 
 
 Degree Days: HDD (Heating Degree Days), CDD (Cooling Degree Days)
 
 Precipitation (in): daily precipitation amount
 
 Snow (in): daily new snowfall, and snow depth
 
 Trace flags: indicators for trace precipitation/snow 

The dataset contains 153 daily records spanning 2025-07-01 to 2025-11-30 (31 + 31 + 30 + 31 + 30 days).

\subsection{Changes, Challenges}
The initial proposal planned to use the Open-Meteo API with hourly observations (temperature, precipitation, humidity, wind). In the final implementation, since we can't find hourly data but only daily data, therefore ,the project used ThreadEx daily climatological summaries for July--November 2025. This change reduced complexity and enabled a clear daily, decision-oriented overview.



\section{Analysis and Visualizations}

\subsection{Analysis Methods}

Time-series analysis: daily temperature and precipitation plotted over time to reveal seasonal trends and short-term variability.
 
Range/variability analysis: the daily minimum-to-maximum temperature band to quantify day-to-night spread.
 
Distribution analysis: a histogram of non-zero precipitation days to characterize typical rain amounts versus rare heavy events.
 
Group-by aggregation: monthly summaries (mean temperature, total precipitation.



\begin{table}[H]
\centering
\caption{Monthly summaries (July--November 2025). Temperatures in \textdegree F; precipitation/snow in inches.}
\label{tab:monthly}
\begin{tabular}{lrrrrrrrr}
\toprule
Month & Days & Avg Temp (mean) & Max Temp (max) & Min Temp (min) & Precip (sum) & Rainy Days & HDD (sum) & CDD (sum) \\
\midrule
2025-07 & 31 & 78.27 & 93 & 62 & 4.66 & 14 & 0 & 420 \\
2025-08 & 31 & 73.02 & 92 & 48 & 0.61 & 4 & 10 & 266 \\
2025-09 & 30 & 69.77 & 89 & 46 & 1.96 & 6 & 15 & 165 \\
2025-10 & 31 & 57.24 & 85 & 34 & 3.88 & 6 & 264 & 29 \\
2025-11 & 30 & 42.87 & 67 & 19 & 2.32 & 9 & 655 & 0 \\
\bottomrule
\end{tabular}
\end{table}


\begin{figure}[H]
    \centering
    \includegraphics[width=0.5\linewidth]{01_daily_avg_temperature.png}
    \caption{Enter Caption}
    \label{fig:placeholder}
\end{figure}

Daily average temperature over time. X-axis is date; Y-axis is temperature (F). 


\begin{figure}[H]
    \centering
    \includegraphics[width=0.5\linewidth]{02_daily_temp_range.png}
    \caption{Enter Caption}
    \label{fig:placeholder}
\end{figure}

Daily minimum-to-maximum temperature range. The shaded band represents (Min, Max) for each day.



\begin{figure}[H]
    \centering
    \includegraphics[width=0.5\linewidth]{03_daily_precipitation.png}
    \caption{Enter Caption}
    \label{fig:placeholder}
\end{figure}
Daily precipitation as a bar chart. Each bar corresponds to one day; spikes indicate heavy-precipitation events.



\begin{figure}[H]
    \centering
    \includegraphics[width=0.5\linewidth]{04_precip_hist_nonzero.png}
    \caption{Enter Caption}
    \label{fig:placeholder}
\end{figure}

Histogram of precipitation amounts on non-zero precipitation days. 



\begin{figure}[H]
    \centering
    \includegraphics[width=0.5\linewidth]{05_monthly_hdd.png}
    \caption{Enter Caption}
    \label{fig:placeholder}
\end{figure}
Monthly Heating Degree Days (HDD). Larger values indicate stronger heating demand.




\begin{figure}[H]
    \centering
    \includegraphics[width=0.5\linewidth]{06_monthly_cdd.png}
    \caption{Enter Caption}
    \label{fig:placeholder}
\end{figure}

Monthly Cooling Degree Days (CDD). Larger values indicate stronger cooling demand. 

\subsection{Observations and Conclusions}

Across July--November 2025:
Clear seasonal cooling: average daily temperatures declined from midsummer into late fall/early winter. Many days have near-zero precipitation, while a small number of days contribute large rainfall totals (long-tailed distribution). CDD is highest in July and declines toward zero by November, while HDD rises sharply in October and peaks in November.The precipitation is not evenly distributed across days. Instead, a small number of heavy-rain days produce prominent spikes, while most days have little to no rainfall.

The results tell us that temperature trends support planning for outdoor work  needs as conditions cool rapidly into late fall.the precipitation plots highlight that weather disruption risk is dominated by a few heavy-rain days rather than frequent moderate rain.The transformation of HDD/CDD marks the end of the need for cooling and the beginning of the need for heating in society and among people.

\section{Future Work}
Given more time, the project could be improved by:Combining years of meteorological data, calculate temperature indicators (heat index, wind chill), and more directly assess the risk days of hot/humid or cold/windy conditions.
Moreover, if one can have data from many years ago, it is possible to calculate and predict the temperature changes and snowfall in the coming month.Finally, the regression relationship between the variation of snowfall and temperature as well as rainfall can be explored



\end{document}
